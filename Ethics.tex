%!TEX root = main.tex

An area of concern that may be a legal, social and ethical issue is whether the users of a website should be informed that their IP address is being collected for analysis and be aware of its repercusions, as collecting IP addresses could lead to individuals being identified, although the database used for the location of an IP states that 'IP geolocation is inherently imprecise. Locations are often near the center of the population' (\cite{maxEthics}). However most privacy policies on websites states that the IP will be collected to analyse trends, including one of the websites that has a agreed to donate data (\cite{PetersWebPrivacy}) and the system will only look at the network and country the IP belongs to.
%When thinking about an ethical way of collecting data one of the key questions is should the website data is collected say the data is being used for this?

%The main ethical issue is around collecting IP addresses this could potentially lead to individual being identified however the system will only look at the network the IP belongs to. 

Whilst the  consent of the website owners were obtained, the one ethical issue could be that the research has not recieved consent from each of the individuals whose data will be analysed. However anyone attempting malicious activity on a website may want their data excluded from the analysis. Therefore it will be difficult to prove if the forumla can pick up malicious activity. One social and ethical issue could be that the entire countries are given a risk, however a user in that country may have a legitimate reason to access a website. Therefore, to mitigate the risk of a social issue, the country that an IP belongs to will be given a minimal weight in the overall formula. 

A potential issue with all security both digital and physical is that people will always try and circumvent the measures put in place, therefore people might try to reverse engineer the analytics. This is not unique to this research as \citeauthor{708447} states that "no software is secure against reverse-engineering."(\cite{708447})
%maybe cite?

A full ethics form for this research can be found in appendix \ref{ethics}

%One wider ethical issue of software like this could be if an attacker was able to figure out the formula they could work out how to get around the formula however that is a potential issue with all security, both digital and physical