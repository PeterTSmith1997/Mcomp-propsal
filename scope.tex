%!TeX root=main.tex
The scope of this work is to build a small programme to analyse the data in website log files to determine if attacks can be detected. The main points of the work are:
\begin{itemize}
    \item identify how attack are evading current techniques 
    \item understand attack characteristics and update the fomula
    \item develop a formula that can detect and determine risk
\end{itemize}

This work will not automatically block IP addresses from accessing the website as that this may cause IP addresses to be incorrectly blocked \citeauthor{TargetedCyberSecurity}, pointed out that the unique 'human' ability to appraise the contextual features of a potential threat means that removing them from the loop of a security methodology is inadvisable. (\cite{TargetedCyberSecurity} ) So this work should be seen as a way to aid the decision making of website owners rather than make the decision for them. 

This work will not check the ability of people to use the software due to the fact it may be difficult to determine if it was the formula or user that identifies attack Bryman. states that "If we suggest that X causes Y, can we be sure that X is responsible for the variation for the Y and not something else". (\citeauthor{bryman_2016} \citeyear{bryman_2016};41). The work is fouced on proving if a formula can idenify attack traffic.

This work will not be looking to generate its own data as it may be easier to prove the accuracy of the formula on real data sets and  means the formula is written in ways that it will be possible to interpret real data.