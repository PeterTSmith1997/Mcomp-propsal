
The largest amount of research done into low rate Dos attacks is by \citeauthor{Adi2015}. His team's work started in 2015 and they have 3 papers on this subject, the last of which was written in 2018. The majority of their work looked at using resource utilisation in order to detect low bandwidth attacks. They set numerous tests to analyse the behaviours of victim machines when subjected to low rate Dos Attacks. 

The 2016 study carried out four varying investigations to analyse the behaviour of a victim machine when subjected to large volume low rate DoS attacks. The researchers used a flood of windows PING and WINDOWS UPDATE frames to Simulate a Low-Rate DoS attack on a server, and demonstrated legitimate HTTP/2 high volume traffic (flash Crowd) could be launched to create a DoS scenario. The team noted at the time of their 2016 investigation that there was no reported study to ascertain whether or not attack traffic could be concealed in a stealthy manner to appear as legitimate flashcrowd traffic. The research concluded that the HTTP/2 protocol itself does not restrict the intensity of traffic generated, and that auxiliary mechanisms should be implemented for identifying volumes and patterns of network traffic \input{Apdenix/TableAdi.tex} (\cite{Adi2016}). Tripathi stated as criticism in his 2018 study that neither the 2015 or 2016 studies completed by Adi's team managed to achieve a completed exhaustion of computational resource (\cite{tripathi2018slow}). Therefore it could be argued that Adi's research did not represent the full effect of a DoS attack. When looking into the details of the results from Adi's research, the figures quoted in Table 1 (Figure 2.1) in the paper by Adi show depletion rates of between 88.39\% to 98.56\% (\cite{Adi2016}).

Due to the ineffective depletion of the CPU the majority of Adi's detection methods could be questioned. Erwin Adi himself points out that his methodology could potentially be flawed due to the reduction of resource status that was achieved. Therefore Adi's proposed method for monitoring CPU depletion as a way to detect low rate bandwidth attacks could be fundamentally in-concise. Adi's study was completed on a server with one single website and did not run any other background services for example, email, which would in theory add unpredictable CPU loads and make his detection method difficult to implement.

\input{Apdenix/HTTP2new.tex}
Although Adi's 2016 work looked at HTTP/2 it was noted in their later 2017 paper that 90\% of contemporary web servers up to the date of that study had not yet migrated to HTTP/2 from HTTP/1.x (\cite{adi2017stealthy}) At the time of this thesis, November 2019, HTTP/2 was used by 41.7\% of the top 10 million websites (\cite{w3techs}), as seen in figure \ref{web http2}. This illuminates the pressing need for research into the safety of the HTTP/2 protocol and consequent detection methods and fixes. 
 
Tripathi's 2018 study took a sample of websites and attempted to detect low rate attacks by monitoring benchmarking and measuring the Chi squared (X\textsuperscript{\small2}) differential value between the expected and observed traffic pattern. Tripathi suggests this approach could detect attacks with high accuracy and may lead to future research to assess further HTTP/2 vulnerabilities, thus potentially mitigating these threat vectors with fixes. Tripathi indicated that although his detection method for attack traffic was successful; if HTTP/2 traffic data is encrypted it must then be decrypted before submitting traces to the detector. He suggested that this could be easily achieved with the aid of an intercepting proxy before forwarding to the target website responsible for handling HTTP/2 requests. It must be noted that most large companies are utilising this strategy to intercept traffic coming into their local network. (\cite{tripathi2018slow}). 

Adi's 2017 work looked at some of the stealthier approaches that cyber attacks were using in order to bypass current detection methods. Adi and his team set up two models intended to simulate stealthy low rate DoS attacks which they called 'bots'. The investigation aimed to model attacks whose traffic continually consumed the victims computing resource, while still being stealthy enough to yield some false alarms via the target servers' 'learning' mechanisms. Adi constructed the attack 'bots' with four core factors for experimentation, these were number of threads, number of window\_update, stealthy factors and the delay between successive TCP connections. These two sets of Stealth models were tested against regular flash crowd traffic in an effort to differentiate the pattern. The experiment and subsequent analysis was successful in distinguishing a notable difference in the patterns of the number of packets carrying SYN flags per 1-second traffic instance between regular flashcrowd traffic and the simulated stealthy attacks (\cite{adi2017stealthy}). Since Adi's work there appears to be an implementation of the methodology proposed in their paper. For example, Cloudflare have been able to implement a defence mechanism against a SYN flood attack. The have created a program called 'gatebot' which monitors SYN packet requests and attempts to drop malicious SYN packets on the firewall layer (\cite{CFSYN}) the work does not refer to that of Adi's however the Cloudflare implementation does appear to use similar concepts. At the  time of writing this seems to be the only documented attempt of blocking SYN attacks.

This section has highlighted the lack of research and mitigation of low rate bandwidth attacks which can easily occur and disrupt a website whilst going undetected.