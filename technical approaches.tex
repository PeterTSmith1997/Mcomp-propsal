%!TeX root=main.tex
The approaches used in this work will be combining existing techniques in a new way. There are websites online that track ip adresses that are attacking websites (such as https://www.abuseipdb.com/) however as they put the ip addresses online, attackers can see when the ip address has been detected. This work intends to hide this infomation from users also, only 10\% of the risk comes from previous knoweldge. There will be a backend database that will hold data about the IP adresses of known bots so that these can return a risk of 0. 

Another table will hold risks of http status codes as  seen in appendix B, by keeping the values in the database table, it is easier to modify the risk. Furthermore another table holds signatures known attacks so that this can be run against the data. This makes it very quick to add new attacks to the formula.

The data analysed will be the apache common log format, this format contains a lot of technical information that will be beneficial to the formula. This data is stored in various data structures that can then be accessed to determine risk. In addition to this, the data from the database, for example the fragments of known bots are getting stored so that they can be cross referenced.  

To be able to identify the country of origin IP address, the Maxmind geoip database is used. Then each country is assigned risk wieghting within a seperate database. The same process is used for the network risk by using the maxmind databases, it ensures that the IP information is always up to date.