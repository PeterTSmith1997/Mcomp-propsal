%!TeX root=main.tex
In a study by \citeauthor{9016229} in 2020, they state that "Low-rate Denial of service (LDoS) attacks has become one of the biggest threats to the Internet, cloud computing platforms, and big data centers" (\cite{9016229}) showing the need for an effective attack detection tool. Studies by Adi (summary here) however \citeauthor{9016229} notes that "Moreover, the detection judgement and defence decision can be achieved by analysing unknown patterns and correlations in network traffic, and grasping other relevant intrinsic information in network traffic."  (\cite{9016229}) this study aims to do that.


The method proposed in the aims, has not been documented as a approach to detect attacks, therefore there is a lack of literature so the review will focuses on gaps in knowledge and where this work fits within that.


Erwin Adi has done a lot of research into Low-rate Denial of Service (LDoS) attacks. His primary paper looks at CPU depletion as an indicator of attack. In the same paper, Adi himself admits that this maybe a flawed technique for attack detection. \cite{Adi2016} 

Most previous studies into detecting Low Bandwidth attacks only look at a single data point such as CPU. This research proposes to look across multiple data points to detect attacks. Furthermore Staniford, Hoagland and McAlerney suggest that storing large amounts of network traffic may be impractical \cite{staniford2002practical}. However\citeauthor{9016229} states that "A huge amount of network traffic can be collected, stored, organized and classified by big data analysis. Moreover, the detection judgement and defense decision can be achieved by analyzing unknown patterns" \cite{9016229} Therefore if there is a need for a large amount of data, that may be impractical to store. One solution may be to look at the data already available to analyse and designing a suitable way to analyse that data.

All the research done to date looks at traffic flow in various ways, however it fails to take into account where that traffic is coming from, for example, most cyber attacks emanate from Russia and China, so the research is ignoring a key area that needs to be explored.

\subsection{revlantcy}
The work is relevant due to the increasing number of websites increasing as nearly all businesses have a website. In the aftermath of remote working and lock downs E-commerce sites increased (CITE NEEDED) the types of attacks the study aims to detect can be hard to identify. Cloudflare notes in August  2022 that 

attacks up on websites