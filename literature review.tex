The method proposed in the aims, has not been documented as a approach to detect attacks, therefore there is a lack of literature so the review will focuses on gaps in knowledge and where this work fits within that.

\textit{High rate attacks on websites can be easily detected and migated.  In the modern world, businesses will take the networking approach of either using a reverse proxy via a provider such as cloudflare, or by the use of in house networking mechanics such as a Software Defined Network (SDN) or a Network Functions Virtualization (NFV). The majority of the papers found while researching high bandwidth attacks are looking into detection methodology while using SDN networking. SDN networking is more important now due to businesses needing highly reliable networks.  As they will now define their network at a software level rather than a hardware level, this will allow them to add or take away capacity from within their network to cope with the peaks in traffic. The downside to this is often that, when not managed correctly, the network will continue to scale up its resources, therefore, increasing the cost to the organisation (\cite{Techbeacon}). 
Haopei Wang notes in 2015, when the emergence of SDN networking was relatively new, that SDN abstracts the physical layer from the hardware layer. Therefore, the network is controlled at the software level; this leads to a more fine grained control of the network and opens up the opportunity for new defence mechanisms. The team developed a DDoS identifying and mitigation process that they named FLOODGUARD. The process involves a two stage saturation mitigation technique involving a Proactive flow rule analyzer and a packet migration module. Through a comprehensive detection algorithm at the migration agent, looking at Memory depletion and rate of packet\_in messages, malign traffic is separated from genuine traffic and mitigated using handling dynamics (\cite{7266854}).
The most recent paper available on the topic was written by \citeauthor{ahalawat2019entropy} He assessed the use of SDN networking and theorised that entropy could be used as a measurement of randomness in an effort to detect DDoS traffic. A mitigation technique was introduced by limiting the rate of packet flow that was allowed to flow to the switch, while in turn, attack control plane bandwidth is prevented by limiting the inflow rate to the controller.  The team concluded that SDN networks seemed, in particular, most vulnerable to DDoS attacks that flooded the network with UDP packets. The researchers noticed that this mitigation technique had the ability to be implemented much earlier than some other techniques that had been previously researched and developed. It was also successful in restoring the use of services to 'normal traffic.' (\cite{ahalawat2019entropy}). Another study to utilize entropy as an early detection and mitigation technique took place in \citeyear{kumar2018safety} achieving similar results (\cite{kumar2018safety}). Although \citeauthor{ahalawat2019entropy} and \citeauthor{kumar2018safety} do not use log file anlyis they show one of the  attack sigins is high traffic rate, but slow rate attacks can cause as much damage if allowed to continue.
}
\subsection{revlantcy}
The work is relevant due to the increasing number of websites increasing...

attacks up on websites